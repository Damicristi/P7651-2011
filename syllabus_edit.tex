\documentstyle[12pt]{article}
\textwidth 6.5in \oddsidemargin 0in \evensidemargin 0in \textheight
8.6in \topmargin -0.5in \pagestyle{empty}
\begin{document}

\vspace*{-1cm}
\begin{center}
{\LARGE \bf Relativistic Quantum Field Theory I}

\vspace*{0.5cm} {\Large Physics 7651}

\vspace*{0.5cm}
\end{center}
\noindent {\bf Time}:  Monday, Wednesday 8$^{30}$-10$^{00}$ am,
location: Rockefeller 105 \vspace*{0.5cm}

\noindent {\bf Lecturer}: Csaba Cs\'aki, 469 Physical Sciences Building, 4-8935,
{\tt csaki@cornell.edu} \vspace*{0.5cm}

\noindent {\bf Content}: This is the first semester class in quantum field theory geared towards physics graduate students.  We are planning to cover the following topics:

\begin{itemize}\addtolength{\itemsep}{-0.5\baselineskip}
\item Free spin-0 fields (Canonical quantization, causality, symmetries)
\item Interacting spin-0 fields (S-matrix, Feynman rules, unitarity, renormalization, spectral decomposition)
\item Spin-1/2 fields (Lorentz and Poincar\'e groups, Weyl fermions, Dirac fermions, quantization and renormalization of spinors) 
\item Functional Integrals 
\item Quantum Electrodynamics (Free vector, gauge invariance, FP Ansatz, QED Feynman rules, elementary processes, dimensional regularization, running coupling, infrared divergences) 
\end{itemize}
 

\noindent {\bf Textbooks}:  Peskin  and Schroeder: \textit{An Introduction to Quantum Field Theory}. We will also use lecture notes by Preskill 
{\tt http://www.theory.caltech.edu/ $\tilde{\hspace*{0.2cm}}$preskill/notes.html} and by Coleman.  Other good books include Zee, Srednicki, and the more encyclopedic Weinberg books. 

 \vspace*{0.5cm}

\noindent {\bf Prerequisites}: Advanced quantum mechanics, some knowledge of special relativity. 

\vspace*{0.5cm}

\noindent {\bf Course requirements}: There will be weekly problem
sets, about 10-12 in total. Problem sets will be posted on Wednsedays
 and will be due in class the next Wednesday. There will be a short take-home final in December. \vspace*{0.5cm}

\noindent {\bf Grades}: The final grade will be determined by
$0.75 \times$ homework $+0.25\times$ final. Undergraduates must take this class for a letter grade. Grad students may opt for pass/fail, but prospective particle theorists are recommended to sign up for a letter grade. 
  \vspace*{0.5cm}



\noindent {\bf Office hour}: Monday 10-11am \vspace*{0.5cm}

\noindent {\bf Grader}: Flip Tanedo (pt267@cornell.edu). Office hour: Tuesday 4:30-5:30pm (tentative), informal section Friday: 11am-noon (tentative). (Flip will be away until 9/5/11) \vspace*{0.5cm}

\noindent {\bf Website}: {\tt blackboard.cornell.edu}, search for Relativistic Quantum Field Theory I  (PHYS7651-CSAKI-FALL2011).
Please enroll on blackboard as a student for e-mail announcements. No more handouts!
% \vspace*{0.5cm}





\end{document}
