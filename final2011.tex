\documentclass[12pt]{article}
\usepackage{amsmath}
\usepackage{amsfonts}
\usepackage{amssymb}
\usepackage{subfig}
\usepackage{graphicx}
\usepackage{float}
\usepackage{bbm}
\usepackage{framed}
\usepackage{setspace}
\usepackage{slashed}

\newcommand{\dbar}{d\mkern-6mu\mathchar'26} 


\textwidth 6.5in
\oddsidemargin 0in
\evensidemargin 0in
\textheight 8.6in
\topmargin -0.7in
\pagestyle{empty}
\begin{document}


\vspace*{-1cm}
\begin{center}
{\LARGE \bf Relativistic Quantum Field Theory}

\vspace*{0.5cm}
{\Large Physics 7651, Fall 2011} \\
\vspace*{0.5cm}
{\Large {\bf Final Exam}\\
\vspace*{1cm}
%Due: To Flip (PSB 432, or Physics Office Mailbox) by December 9}
Please read the following instructions carefully}
\end{center}

% \begin{framed}
% \noindent You have 24 hours to complete this exam. The \textit
% \begin{enumerate}
% 	\item The last day for homeworks to be submitted for grading is \textbf{December 9th}
% 	\item You are \textit{not} allowed to work on this homework concurrently with the 24-hour take home final exam. The exam will include material up to the last class, you are encouraged to complete this homework before the exam.
% 	\item There will be an office hour on \textit{Monday} December 5th from 4pm -- 5:30pm to discuss this homework and any questions about the course.
% \end{enumerate}
% \end{framed}

\vspace{.25cm}

\begin{spacing}{.55}
\begin{enumerate}
\item You have 24 hours to complete this exam. 
\item Upon completion, please staple all pages together, place the exam back in its envelope, and return it directly to Flip in PSB 432.
%
\item The \textit{only} references allowed are Peskin \& Schr\"oeder, the lecture notes by Preskill and Coleman, your homework, homework solutions, and integral tables like Gradshteyn \& Ryzhik. (Official solutions may have minor errors---you are responsible for correcting them if you use these results!) If you use one of these sources, please provide a reference.
\item You may \textit{not} consult outside references `for your homework' (e.g. Homework 13) while you are working on this exam.
\item You may \textit{not} collaborate on this exam, not even with Google / Wikipedia / Siri or any other imaginary friends that you may have made while taking this course.
\item \textbf{Write clearly and box your final answers}. We cannot give partial credit if we don't know where you went wrong. Trivial algebraic/arithmetic slips will be penalized much less than fundamental misunderstandings.
\item There will be loop integrals on this exam which you will be asked to calculate using a hard cutoff. You may solve these by hand, appealing to an integral table, or using computer software. You should provide an appropriate reference for any result.
\item If you require clarification, contact Flip at \texttt{pt267@cornell.edu}.
\end{enumerate}
\end{spacing}
\vspace{.7 cm}

\textsc{Please read and sign below}

I have read and understand these rules to this exam and agree to abide by them.
\vspace{2em}

\line(1,0){250} 

\textsc{Signature and Date}

\vspace{2em}

\line(1,0){250} 

\textsc{Print name and Student ID number}

\newpage

\begin{enumerate}
	
	
	\item {\bf Pion decay} [50 points]
	
	The dominant decay mode of the $\pi^+$ (pion) is into a $\mu^+$ (antimuon) and $\nu_\mu$ (muon neutrino). Assume that the only interactions of these particles is described by
	\begin{align*}
		\mathcal{L}_\text{Int} = (\partial_\lambda \pi) \bar\nu \gamma^\lambda (f+g\gamma_5)\mu + \text{ complex conjugate}.
	\end{align*}
	The muon and neutrino are Dirac fermions, the couplings $f$ and $g$ are complex numbers.
	
	\begin{enumerate}
		\item Compute the decay width $\Gamma(\pi^+\to \bar\mu + \nu_\mu)$ to second order in the coupling constants (the lowest non-vanishing order) in terms of the coupling constants, the pion mass $m_\pi$, and the muon mass $m_\mu$. Assume the neutrino mass is zero. Assume the validity of the na\"ive Feynman rules for derivative interactions as explained in the homework.
		\item  The neutrinos emitted in this process are experimentally observed to always have helicity $-1/2$. What does this imply for the ratio $f/g$?
		\item A much less frequent decay mode of the $\pi^+$ is into a positron and an electron neutrino. Assume that this is caused by an interaction of the same form as above with the following replacements:
		\begin{itemize}
			\item Electron and electron neutrino fields replace the muon and muon neutrino
			\item Couplings $f'$ and $g'$ replace $f$ and $g$
		\end{itemize}
		Electron neutrinos, like muon neutrinos, are massless and always have helicity $-1/2$. Use the following data to compute $|f'/f|$ to within 10\% accuracy:
			\begin{align*}
				m_\pi &= 140 \text{ MeV}\\
				m_\mu &= 106 \text{ MeV}\\
				m_e &= 0.51 \text{ MeV}\\
				\frac{\Gamma(\pi^+\to \bar e \nu_e)}{\Gamma(\pi^+\to \bar \mu \nu_\mu)}
				&= 1.23 \times 10^{-4}.
			\end{align*}
	\end{enumerate}
	

\vspace{.5em}


\item {\bf Renormalization of Yukawa theory} [50 points]

In Homework 13 you renormalized the two-point function in pseudoscalar Yukawa theory. In this problem we'll finish the job. 
%\begin{align}
%\mathcal L = \bar\Psi_0 (i\gamma^\mu-M_0) \partial_\mu \Psi_0 + \frac 12 (\partial\phi_0)^2 - \frac 12 m_0^2\phi_0^2-y_0\phi_0\bar\Psi_0 \Psi_0 + \frac{\lambda_0}{4!}\phi_0^4.
%\end{align}
\begin{align*}
\mathcal L = \bar\Psi (i\gamma^\mu-M) \partial_\mu \Psi + \frac 12 (\partial\varphi)^2 - \frac 12 m^2\varphi^2-iy\varphi\bar\Psi\gamma^5 \Psi + \frac{\lambda}{4!}\varphi^4.
\end{align*}
%You may assume that $\Pi_\Psi(p^2)$ and $\Pi_\phi(p^2)$ have been calculated to the appropriate order in perturbation theory so that $Z_\Psi$, $Z_\phi$, and the mass counter-terms $\delta_M$ and $\delta_{m^2}$ are known. (You did them correctly in the homework, right?)
We shall use renormalization conditions that fix $y$ and $\lambda$ when all external momenta vanish. (This will simplify your loop calculations.) Determine the counter-terms for the interaction terms, $\delta_y$ and $\delta_\lambda$, at one-loop order.



%	\item {\bf Effective theory of a composite scalar} [55 points]
%	
%	In this problem we will study the theory of a complex scalar that is formed as a fermion--anti-fermion bound state. We will remain agnostic about the high-energy dynamics that generate the bound state and will describe the low-energy behavior using an effective theory. The theory at the high scale $\Lambda_P$ is written in terms of the Dirac fermion $\Psi$,
%	\begin{align}
%%		\mathcal L = \chi^* i\partial_\mu\bar\sigma^\mu\chi + \psi^* i \partial_\mu\sigma^\mu \psi + G(\chi^*\psi)(\psi^*\chi).
%\mathcal L = \bar\Psi i\gamma^\mu \partial_\mu\Psi + G(\bar\Psi P_L \Psi)(\bar \Psi P_R \Psi),
%		\label{eq:UV}
%	\end{align}
%	where $P_L = \frac 12 (1-\gamma^5)$ and $P_R = \frac 12 (1+\gamma^5)$ are chiral projection operators and $\bar\Psi = \Psi^\dag \gamma^0$. Note that $\Lambda_P$ is a \textit{physical} scale rather than an unphysical cutoff.
%	\begin{enumerate}
%		\item What is the dimension of the coupling $G$? The interaction term can be understood as coming from the exchange of a gauge boson of mass $\Lambda_P$ and momentum $q^2\ll \Lambda^2$. The amplitude for such an interaction (as you will see next semester) is
%		\begin{align}
%			\frac 12 g^2 \left(\bar\Psi \gamma^\mu P_L \Psi\right)
%			\frac{g_{\mu\nu}}{q^2-\Lambda_P^2} 
%			\left(\bar\Psi \gamma^\nu P_R \Psi\right).
%			\label{eq:exchange}
%		\end{align} 
%		In the limit where $q^2 \ll \Lambda_P$, use `Fierz identities' to show that (\ref{eq:exchange}) reduces to (\ref{eq:UV}). What is $G$ in terms of $g$ and $\Lambda$ in this limit? \textbf{Hint:} it may be easier to write this using Weyl spinors and use the Pauli matrix identities from Homework 12. \textbf{Remark:} the validity of (\ref{eq:UV}) as an effective theory of (\ref{eq:exchange}) is not obvious; for the purposes of this exam take it as an assumption.
%	\end{enumerate}
%	We will explore this theory at scales much lower than $\Lambda_P$. As a trick, we will modify the Lagrangian by writing the composite scalar $\phi_0$ as a non-dynamical `auxiliary field' at the scale $\Lambda_P$. Renormalization effects coming from loops of the fermions will generate dynamics for $\phi_0$.
%	\begin{enumerate}
%		\item[(b)] We rewrite (\ref{eq:UV}) in terms of an auxiliary field $\phi_0$:
%		\begin{align}
%			\mathcal L = \bar\Psi i\gamma^\mu \partial_\mu\Psi + (g \phi \bar\Psi P_L \Psi  + g\phi^*\bar\Psi P_R\Psi) - \Lambda_P^2 \phi^* \phi.
%			\label{eq:auxiliary}
%		\end{align}
%		At this stage, note that $\phi_0$ is a strange field that has no kinetic term. This means that it is non-dynamical and can be solved algebraically through its equation of motion. Explicitly state the equation of motion for $\phi_0$ and then plug this back into (\ref{eq:auxiliary}) to show that it indeed reproduces (\ref{eq:UV}).
%		\end{enumerate}
%		In order to incorporate the dynamics of the composite scalar, we write the Lagrangian in the following form:
%		\begin{align}
%			\mathcal L =& \bar\Psi i\gamma^\mu \partial_\mu\Psi + (g \phi_0 \bar\Psi P_L \Psi  + g\phi_0^*\bar\Psi P_R\Psi)
%			%\nonumber \\&
%			+ Z_0 |\partial \phi_0|^2 - m_0^2 \phi_0^*\phi_0 - \frac{\lambda_0}{2}|\phi_0^*\phi_0|^2.
%			\label{eq:effective}
%		\end{align}
%		At the scale $\Lambda_P$, we choose renormalization conditions such that (\ref{eq:effective}) matches (\ref{eq:auxiliary}). For example, at tree level, $m_0^2 = \Lambda_P^2$ and $Z_0 =\lambda_0=0$. We will see shortly that loops generate non-zero contributes to these terms so that $\phi_0$ becomes a dynamical field. Thus, at scales $\mu<\Lambda_P$, we can treat $\phi_0$ as if it were a fundamental complex scalar in our theory and calculate using (\ref{eq:effective}). We will now renormalize the $\phi_0$ sector of the theory in the ``fermion bubble approximation'' wherein we only consider loops that only contain fermion lines and ignore any loop with an internal scalar line. \textbf{Remark:} for the purposes of this exam, do not worry about the validity of this approximation.
%		\begin{enumerate}
%		%%%
%		\item[(c)]		Draw the loop diagrams that contribute to the scalar 2-point and 4-point functions. Calculate the loop diagrams for the 1PI correction to the propagator and $2\to 2$ scattering.  Assume that all external states have momentum $p$. Separate the bare Lagrangian into a renormalized Lagrangian and a counter-term Lagrangian with respect to the scalar sector parameters,
%		\begin{align}
%			\mathcal L =& \cdots
%			%\bar\Psi i\gamma^\mu \partial_\mu\Psi + (g \phi_0 \bar\Psi P_L \Psi  + g\phi_0^*\bar\Psi P_R\Psi)
%			%\nonumber \\&
%			+ (Z+\delta_Z) |\partial \phi_0|^2 - (m^2+\delta_{m_0^2}) \phi_0^*\phi_0 - \frac{1}{2}(\lambda+\delta_{\lambda_0})|\phi_0^*\phi_0|^2.
%			\label{eq:counterterms}
%		\end{align}
%		Recall that $\delta_Z, \delta_{m_0^2} \sim \mathcal O(g^2)$ and $\delta_\lambda \sim \mathcal O(g^4)$. These counter-terms are defined such that when $p^2=\Lambda_P^2$, (\ref{eq:effective}) matches (\ref{eq:auxiliary}). They should depend on both the unphysical cutoff $\Lambda$ (regularizing any divergences) and the renormalization scale $\Lambda_P$. The $\Lambda$ dependence should cancel in physical observables. Note that we have not yet canonically normalized the field $\phi_0$. What are the one-loop values of $Z$, $m'^2$, and $\lambda'$ at a scale $\mu <\Lambda$?
%		\item[(d)] Finally, write down the physical couplings at the scale $\mu$ by canonically normalizing the fields. Write out the values of the physical coupling constants, $g_P = g^2/Z$, $m^2_P = m^2 / Z$, etc. \textbf{Remark:} we never calculated any loops to correct the Yukawa coupling $g$, but it is still rescaled at this step. 
%		%%%

%	\end{enumerate}

\end{enumerate}

\noindent All done? \textit{Please} make sure your answers are written legibly with answers boxed. It has been a pleasure having you part of the course. Best wishes with your remaining exams and have a good winter break.



\newpage

\vspace*{-1cm}
\begin{center}
{\LARGE \bf Relativistic Quantum Field Theory}

\vspace*{0.5cm}
{\Large Physics 7651, Fall 2011} \\
\vspace*{0.5cm}
{\Large {\bf Final Exam}\\
\vspace*{1cm}
%Due: To Flip (PSB 432, or Physics Office Mailbox) by December 9}
Please read the following instructions carefully}
\end{center}

% \begin{framed}
% \noindent You have 24 hours to complete this exam. The \textit
% \begin{enumerate}
% 	\item The last day for homeworks to be submitted for grading is \textbf{December 9th}
% 	\item You are \textit{not} allowed to work on this homework concurrently with the 24-hour take home final exam. The exam will include material up to the last class, you are encouraged to complete this homework before the exam.
% 	\item There will be an office hour on \textit{Monday} December 5th from 4pm -- 5:30pm to discuss this homework and any questions about the course.
% \end{enumerate}
% \end{framed}

\vspace{.25cm}

\begin{spacing}{.55}
\begin{enumerate}
\item You have 24 hours to complete this exam. 
\item Upon completion, please staple all pages together, place the exam back in its envelope, and return it directly to Flip in PSB 432.
%
\item The \textit{only} references allowed are Peskin \& Schr\"oeder, the lecture notes by Preskill and Coleman, your homework, homework solutions, and integral tables like Gradshteyn \& Ryzhik. (Official solutions may have minor errors---you are responsible for correcting them if you use these results!) If you use one of these sources, please provide a reference.
\item You may \textit{not} consult outside references `for your homework' (e.g. Homework 13) while you are working on this exam.
\item You may \textit{not} collaborate on this exam, not even with Google / Wikipedia / Siri or any other imaginary friends that you may have made while taking this course.
\item \textbf{Write clearly and box your final answers}. We cannot give partial credit if we don't know where you went wrong. Trivial algebraic/arithmetic slips will be penalized much less than fundamental misunderstandings.
\item There will be loop integrals on this exam which you will be asked to calculate using a hard cutoff. You may solve these by hand, appealing to an integral table, or using computer software. You should provide an appropriate reference for any result.
\item If you require clarification, contact Flip at \texttt{pt267@cornell.edu}.
\end{enumerate}
\end{spacing}

\vspace{.7 cm}

\noindent Please indicate the time and date when you opened and sealed this envelope. 
%
By signing this document you agree that you have completed the exam within the allowed time.
%
Upon completion, please bring this envelope to Flip in PSB 432.
\vspace{2em}

\line(1,0){220} \qquad \line(1,0){220} 

\textsc{Envelope opened (time and date)}
\hspace{1.1cm}
\textsc{Envelope sealed (time and date)}


%\vspace{2em}
%
%\line(1,0){250} 
%
%\textsc{Envelope sealed (time and date)}


\vspace{2em}

\line(1,0){220} 

\textsc{Signature}



\end{document}
