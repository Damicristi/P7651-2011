\documentclass[12pt]{article}
\textwidth 6.5in
\oddsidemargin 0in
\evensidemargin 0in
\textheight 8.6in
\topmargin -0.5in
\pagestyle{empty}
\usepackage{amssymb,amsmath}

\begin{document}

\vspace*{-1cm}
\begin{center}
{\LARGE \bf Relativistic Quantum Field Theory}

\vspace*{0.5cm}
{\Large Physics 7651} \\
\vspace*{0.5cm}
{\Large {\bf Homework 2. }\\
\vspace*{0.5cm}
Due: In on Wednesday, Sept. 14 to Flip Tanedo's mailbox}
\end{center}

{\bf Reminder: make-up class on Friday, Sept. 9 8:3-10 am in Rock 105. No class on Wednseday, Sept. 14}


\begin{enumerate}

\item  {\bf Explicit expression of correlator} 

We have used the correlator 
\[ \langle 0| \Phi (x) \Phi (y) |0\rangle = \Delta_+(x-y)= \int \frac{d^3 k}{(2\pi )^32 \omega_k }e^{-i k\cdot (x-y)} \]
many times in class when discussing causality. Find an explicit expression for $\Delta_+ (r)$ in terms of Bessel functions for space-like $x$ with $x^2=-r^2<0$.  


\vspace*{0.5cm}

\item {\bf Time ordered product as Green's function}

The time-ordered product of two fields $A(x)$ and $B(x)$ is defined by
% \begin{eqnarray}
% T[A(x) B(y)]& =& A(x) B(y) \  \ {\rm for } \  \  x^0>y^0 \nonumber \\ 
% & = & B(y) A(x) \  \  {\rm for } \  \  y^0>x^0 \nonumber 
% \end{eqnarray}
\begin{align*}
	T\left[A(x)B(y)\right] &= \left\{
		\begin{array}{ll}
			A(x)B(y) & \quad \text{for } x^0 > y^0\\
			B(y)A(x) & \quad \text{for } x^0 < y^0
		\end{array}
	\right.
\end{align*}

Using only the field equation and the equal time commutation relations, show that for a free scalar field $\Phi$ with mass $m$,
\[ (\square_x +m^2) \langle 0|T[ \Phi (x)\Phi(y)|0\rangle = c \delta^4 (x-y) ,\]
and find the proportionality constant $c$.

 \vspace*{0.5cm}

\item {\bf Feynman propagator}

Show that 
\[ \langle 0| T[\Phi(x)\Phi(y) ]|0\rangle = \lim_{\epsilon\to 0^+} \int \frac{d^4k}{(2\pi)^4} e^{-ik\cdot (x-y)} \frac{-c}{k^2-m^2+i\epsilon}.\]

The limit symbol indicates that $\epsilon$ goes to zero through positive values. If the $\epsilon$ were not present the integral would be ill-defined, because it would have poles in the domain of integration. Hint: Do the $k^0$ integration and compare with the expression for the left-hand side obtained by inserting a complete set of intermediate states. Here you can use anything we know about the field $\Phi$.
  
\end{enumerate}

\end{document}
