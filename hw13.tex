\documentclass[12pt]{article}
\usepackage{amsmath}
\usepackage{amsfonts}
\usepackage{amssymb}
\usepackage{subfig}
\usepackage{graphicx}
\usepackage{float}
\usepackage{bbm}
\usepackage{framed}

\newcommand{\dbar}{d\mkern-6mu\mathchar'26} 


\textwidth 6.5in
\oddsidemargin 0in
\evensidemargin 0in
\textheight 8.6in
\topmargin -0.5in
\pagestyle{empty}
\begin{document}

\vspace*{-1cm}
\begin{center}
{\LARGE \bf Relativistic Quantum Field Theory}

\vspace*{0.5cm}
{\Large Physics 7651} \\
\vspace*{0.5cm}
{\Large {\bf Homework 13}\\
\vspace*{0.5cm}
Due: To Flip (PSB 432, or Physics Office Mailbox) on December 2}
\end{center}

\begin{framed}
\noindent Please e-mail Flip (\texttt{pt267@cornell.edu}) with any corrections; major corrections will be updated immediately on Blackboard. Please note the following announcements:
\begin{enumerate}
	\item The last day for homeworks to be submitted for grading is \textbf{December 9$^\text{th}$.}
	\item You are \textit{not} allowed to work on this homework concurrently with the 24-hour take home final exam. The exam will include material up to the last class, you are encouraged to complete this homework before the exam.
	\item There will be the usual office hour on Tuesday, November 29$^\text{th}$ and an additional office hour on \textit{Monday} December 5th from 4pm -- 5:30pm to discuss any questions about the course or previous homework.
	\item Flip will not be available from Wednesday, Nov.~30 through Friday Dec.~2.
\end{enumerate}
\end{framed}

\begin{enumerate}

\item {\bf Boosting the $x$ and $y$ spinors} [10 points]

As we have seen in class, the $x$ and $y$ spinors can be chosen in the rest frame to be 
\[ x_\alpha (\vec{p}=0,s)=y^{\dagger\dot\alpha} (\vec{p}=0,s)=\sqrt{m} \chi_s \]
where the $\chi_s$ are the eigenvectors of the spin operator along an arbitrarily chosen axis.
Boost these rest frame spinors to show that in an arbitrary frame,
\begin{eqnarray}
& x_\alpha (\vec{p},s)= \sqrt{p\cdot \sigma} \chi_s , \ \ \ & x^\alpha (\vec{p},s)=(-2s) \chi_{-s}^\dagger   \sqrt{p\cdot \bar\sigma}  \nonumber \\
& y_\alpha (\vec{p},s)= (2s) \sqrt{p\cdot \sigma} \chi_{-s} ,  & y^\alpha (\vec{p},s)= \chi_s^\dagger  \sqrt{p\cdot \bar\sigma}.  \nonumber 
\end{eqnarray}


\item {\bf Trace identities for $\gamma$ matrices} [10 points]

Prove the following useful trace identities for $\gamma$ matrices.
\begin{align}
	\text{Tr} (\gamma^\mu\gamma^\nu) & = 4 g^{\mu\nu}\\
	\text{Tr} (\gamma^\mu\gamma^\nu\gamma^\rho\gamma^\sigma) & = 4 (g^{\mu\nu}g^{\rho\sigma} - g^{\mu\rho}g^{\nu\sigma}+g^{\mu\sigma}g^{\nu\rho})\\
	\text{Tr} (\gamma^\mu\gamma^\nu\gamma^\rho\gamma^\sigma\gamma^5) & = -4i\epsilon^{\mu\nu\rho\sigma}
\end{align}

\item {\bf Yukawa theory at tree level} [15 points]

Consider the theory of a real scalar $\phi$ with a Yukawa coupling, $\Delta \mathcal L_I = -y\phi\bar\Psi \Psi$, to a Dirac spinor $\Psi$. Assume the scalar has mass $m$ and the fermion has mass $M$.

\begin{enumerate}
	\item Draw the $\mathcal O(y^2)$ diagrams for $\Psi\bar\Psi\to \Psi\bar\Psi$ scattering. Calculate the amplitude and cross section for this process.
	\item Draw the $\mathcal O(y^2)$ diagrams for $\Psi\phi \to \Psi\phi$ scattering. Calculate the amplitude and cross section for this process.
	\item Draw the $\mathcal O(y)$ diagram for the decay of a $\phi$ and calculate the decay rate.
	\item Redo the above problems for a real pseudoscalar $\varphi$ with a Yukawa coupling, $\Delta \mathcal L_I = -i\varphi\bar\Psi\gamma^5 \Psi$.
\end{enumerate}

\item {\bf Yukawa theory at loop level} [15 points]

Consider the pseudoscalar Yukawa theory above. 
\begin{enumerate}
	\item Write out the Lagrangian for this theory and show that it is parity invariant.
	\item Determine expression for the superficial degree of divergence as a function of the number of external legs of each type. Classify all superficially divergent diagrams (`blob diagrams') and list their superficial degree of divergence. Use the symmetry of the theory to identify which diagrams must vanish identically.
	\item This is a renormalizable theory. Recall that for such a theory we are instructed to write down all possible renormalizable interactions that are allowed by the symmetries of the theory. What term are we missing in the na\"ive Lagrangian? (\textit{Hint:} you should have seen this in the previous problem and thought, ``hey, we don't have a counter term for that!'')
	\item Draw the loop-level diagrams for each of the superficially divergent processes.
	\item Renormalize the theory and calculate the renormalized 2-point functions and the counter-terms needed to renormalize these 2-point functions. For simplicity, you do not need to calculate 3- and higher-point functions.
	\item Is the 2-point amplitude for $\Psi$ as divergent as superficially expected? (If not, why not? Make an argument based on symmetry.)
\end{enumerate}

%Also, in 4 I would like you to say that they should renormalize the theory and calculate the renormalized 2 point functions, and calculate the counter terms needed to renormalize these 2 pt functions.

\end{enumerate}

\vspace{1em}

\noindent Since this is the last problem set, here's a quote from Coleman's original lectures: 
\begin{quote}
Not only God knows, I know, and by the end of the semester, you will know.
\end{quote}
Happy Thanksgiving, and good luck on the final!

 

\end{document}
